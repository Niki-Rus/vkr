\section{Техническое задание}
\subsection{Основание для разработки}

Основанием для разработки является задание по курсовой работе 
<<Разработка программной библиотеки для шифрования данных по стандарту AES-128>>. 

\subsection{Цель и назначение разработки}

Целью курсовой работы является разработка программной библиотеки для шифрования данных по стандарту AES-128.  

Разработка библиотеки направлена на получение практических навыков проектирования и реализации  алгоритмов, а также на формирование базиса для последующего применения в системах защиты информации.  

Основными задачами данной разработки являются:
\begin{itemize}
	\item изучение функций шифрования
	\item разработка функциональных требований
	\item проектирование библиотеки шифрования 
	\item реализация функций шифрования
	\item тестирование функций шифрования
\end{itemize}

\subsection{Функции библиотеки}

Библиотека должна включать в себя следующие функции:
\begin{itemize}
	\item  функции для шифрования и дешифрования данных по алгоритму AES-128;
	\item функции для приёма входных данных из потока и их передачи на шифрование;
\end{itemize}


%Композиция шаблона сайта представлена на рисунке ~\ref{templ:image}.
%\begin{figure}[ht]
%\includegraphics[width=1\linewidth]{templ}
%\caption{Композиция шаблона сайта}
%\label{templ:image}
%\end{figure}
%\vspace{-\figureaboveskip} % двойной отступ не нужен (можно использовать, если раздел заканчивается картинкой)

\subsection{Описание программн интерфейса}
Расписать функции выходного потока тут

Для разрабатываемого сайта была реализована модель, которая обеспечивает наглядное представление вариантов использования сайта.

Она помогает в физической разработке и детальном анализе взаимосвязей объектов. При построении диаграммы вариантов использования применяется унифицированный язык визуального моделирования UML.

Диаграмма вариантов описывает функциональное назначение разрабатываемой системы. То есть это то, что система будет непосредственно делать в процессе своего функционирования. Она является исходным концептуальным представлением системы в процессе ее проектирования и разработки. Проектируемая система представляется в виде ряда прецедентов, предоставляемых системой актерам или сущностям, которые взаимодействуют с системой. Актером или действующим лицом является сущность, взаимодействующая с системой извне (например, человек, техническое устройство). Прецедент служит для описания набора действий, которые система предоставляет актеру.

На основании анализа предметной области в программе должны быть реализованы следующие прецеденты:
\begin{enumerate}
\item Просмотр информации о компании.
\item Просмотр информации о продукции компании.
\item Просмотр информации об услугах компании, много услуг, длинный список, не поместится никак в одну строку.
\item Поиск по сайту.
\end{enumerate}

\subsection{Требования к оформлению документации}

Разработка программной документации и программного изделия должна производиться согласно ГОСТ 19.102-77 и ГОСТ 34.601-90. Единая система программной документации.
