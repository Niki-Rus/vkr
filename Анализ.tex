% !TeX spellcheck = ru_RU-Russian
\section{Анализ предметной области}
\subsection{Характеристика области и её применения}

В современном цифровом мире безопасность информации является одной из ключевых задач. С каждым годом возрастает объём передаваемых и хранимых данных: от личной переписки и фотографий до банковских транзакций и государственных архивов. Любая утечка, подделка или уничтожение данных может привести к серьёзным экономическим, политическим и социальным последствиям.  

Для защиты информации используются криптографические методы, среди которых центральное место занимает шифрование. Оно обеспечивает три важнейших свойства информационной безопасности: \textit{конфиденциальность}, \textit{целостность} и \textit{доступность}. В современных условиях без применения криптографии невозможно представить работу интернета, электронных платежей, мобильной связи, облачных сервисов, цифровых документов и даже бытовых устройств, объединённых в сети «умного дома».  

Одним из наиболее распространённых инструментов защиты является алгоритм AES (Advanced Encryption Standard), который на сегодняшний день используется как в коммерческих, так и в государственных системах по всему миру.  

\subsection{История и развитие симметричных алгоритмов шифрования}

История криптографии насчитывает тысячелетия. Ещё в античные времена применялись простейшие методы скрытия текста: например, шифр Цезаря, основанный на циклическом сдвиге букв алфавита. Позже появились более сложные системы, такие как шифр Виженера, использовавший повторяющийся ключ. Эти методы были эффективны для своего времени, однако в эпоху развития вычислительной техники они оказались недостаточными.  

С середины XX века началась активная разработка алгоритмов, предназначенных для защиты электронных данных. В 1977 году был принят стандарт DES (Data Encryption Standard), основанный на блочном симметричном шифре с длиной ключа 56 бит. DES стал первым массовым криптографическим стандартом и долгое время использовался в банковских и коммерческих системах. Однако со временем, в связи с ростом вычислительных мощностей, он оказался уязвимым для атак полного перебора.  

Для повышения стойкости в 1990-е годы была предложена модификация 3DES (Triple DES), в которой алгоритм DES выполнялся трижды с разными ключами. Этот метод повысил надёжность, но существенно замедлил работу. В результате стало ясно, что требуется новый стандарт шифрования, который сочетал бы безопасность и высокую производительность.  

\subsection{Конкурс NIST и появление AES}

В 1997 году Национальный институт стандартов и технологий США (NIST) объявил международный конкурс на разработку нового алгоритма симметричного шифрования, который должен был заменить устаревший DES. К алгоритму выдвигались строгие требования: высокая криптографическая стойкость, эффективность на различных аппаратных платформах, гибкость в выборе длины ключа, а также простота реализации.  

На конкурс было представлено 15 кандидатов со всего мира, среди которых выделялись такие алгоритмы, как MARS (IBM), Serpent, Twofish и Rijndael. После нескольких лет тестирования, анализа и открытого обсуждения криптографическим сообществом, в 2000 году победителем был признан алгоритм Rijndael, созданный бельгийскими учёными Винсентом Рейменом и Жоаном Дайменом.  

В 2001 году Rijndael был официально утверждён под названием AES (Advanced Encryption Standard). С этого момента он стал новым международным стандартом симметричного шифрования и используется по сей день.  

\subsection{Классификация методов шифрования}

Современные криптографические методы можно разделить на несколько категорий:  

\begin{itemize}
	\item \textbf{Симметричные алгоритмы} — используют один и тот же ключ для шифрования и дешифрования. Их преимущество заключается в высокой скорости работы, что делает их незаменимыми для обработки больших массивов данных. К этой категории относятся AES, DES, 3DES, RC4.  
	\item \textbf{Асимметричные алгоритмы} — применяют пару ключей: открытый для шифрования и закрытый для расшифровки. Такие системы более удобны для организации защищённого обмена ключами, но работают медленнее. Примеры: RSA, ElGamal.  
	\item \textbf{Гибридные системы} — объединяют преимущества двух предыдущих подходов: асимметричное шифрование используется для безопасной передачи ключа, а симметричное — для дальнейшей быстрой работы с данными. Примером является протокол TLS, лежащий в основе защищённых интернет-соединений (HTTPS).  
\end{itemize}

Алгоритм AES относится к симметричным блочным шифрам. Он работает с блоками фиксированной длины 128 бит и поддерживает три варианта длины ключа: 128, 192 и 256 бит. В зависимости от длины ключа выполняется различное число раундов: 10, 12 или 14 соответственно. Каждый раунд включает четыре последовательных преобразования: замену байтов (SubBytes), циклический сдвиг строк (ShiftRows), линейное преобразование столбцов (MixColumns) и наложение раундового ключа (AddRoundKey).  

\subsection{Математические основы AES}

Одним из ключевых достоинств AES является строгое математическое обоснование его устойчивости. Все операции внутри алгоритма выполняются не в обычной арифметике, а в специальной алгебраической структуре — поле Галуа $GF(2^8)$.  

Поле $GF(2^8)$ представляет собой множество из 256 элементов, где каждая величина соответствует одному байту (8 битам). Сложение в этом поле выполняется по модулю 2 (операция XOR), а умножение задаётся с использованием фиксированного неприводимого многочлена степени 8. Такая организация позволяет эффективно описывать все необходимые преобразования с помощью простых побитовых операций.  

Главным нелинейным преобразованием в AES является \textbf{таблица подстановок (S-box)}, которая строится на основе обратных элементов в поле $GF(2^8)$ с последующим аффинным преобразованием. Это обеспечивает высокую криптографическую стойкость за счёт отсутствия простых зависимостей между входом и выходом.  

Операция \textbf{ShiftRows} реализует циклический сдвиг строк, что разрушает линейные зависимости внутри блока.  
Операция \textbf{MixColumns} интерпретирует каждый столбец блока как многочлен над $GF(2^8)$ и выполняет его умножение на фиксированный многочлен. Это обеспечивает сильное перемешивание байтов и равномерное распространение изменений по всему блоку.  
Операция \textbf{AddRoundKey} добавляет к состоянию блока раундовый ключ с помощью операции XOR.  

Таким образом, каждая стадия AES основана на строгих математических принципах: нелинейность достигается через использование обратных элементов поля, а диффузия (распространение изменений) обеспечивается линейными преобразованиями над многочленами.  

\subsection{Значение и применение AES}

Сегодня AES является основным международным стандартом шифрования. Он применяется во множестве современных технологий и стандартов:  

\begin{itemize}
	\item в протоколах защиты интернет-соединений (HTTPS, TLS, VPN);  
	\item при шифровании беспроводных сетей (WPA2, WPA3);  
	\item в банковских системах, электронных платежах и банкоматах;  
	\item в мобильных устройствах, SIM-картах и мессенджерах;  
	\item в государственных и военных системах связи;  
	\item в архивных и файловых форматах (например, ZIP с AES-шифрованием).  
\end{itemize}

За более чем двадцать лет применения AES не было найдено практических криптоанализов, способных разрушить его безопасность. Все существующие атаки ограничиваются теоретическими исследованиями или требуют нереалистичных вычислительных ресурсов. На практике AES остаётся надёжным и эффективным инструментом обеспечения конфиденциальности данных.  

Таким образом, алгоритм AES сочетает в себе устойчивость к атакам, высокую производительность и универсальность применения. Его использование является обязательным условием для построения современных систем информационной безопасности.  
