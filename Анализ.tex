% !TeX spellcheck = ru_RU-Russian
\section{Анализ предметной области}
\subsection{Характеристика области и её применения}

С ростом объёмов информации в конце XX – начале XXI века особое значение приобрели технологии эффективного хранения и передачи данных. Наиболее важным направлением в данной области стали алгоритмы сжатия информации. Их основная цель заключается в уменьшении избыточности представления данных при сохранении возможности полного восстановления исходного сообщения. 

Применение методов сжатия позволяет значительно снизить объём данных, что в свою очередь уменьшает требования к дисковому пространству и скорости передачи по сетям. Сегодня алгоритмы сжатия применяются практически во всех сферах цифровых технологий: от архиваторов и текстовых редакторов до графических форматов и потокового мультимедиа. 

\subsection{История и развитие алгоритмов сжатия}

Первые работы по сжатию информации восходят к середине XX века и связаны с исследованиями в области теории информации К. Шеннона. Позднее были разработаны практические алгоритмы, позволяющие уменьшать объём текстовых и бинарных данных. Одним из наиболее известных стал алгоритм LZ77, предложенный Абрахамом Лемпелем и Якобом Зивом в 1977 году. 

На основе его идей в 1984 году Терри Уэлч разработал модификацию под названием LZW (Lempel–Ziv–Welch). Алгоритм быстро получил популярность благодаря своей эффективности и простоте реализации. Он лег в основу многих распространённых форматов: графических (GIF, TIFF), текстовых и архивных (например, Unix-команда \texttt{compress}). 

\subsection{Классификация методов сжатия}

Алгоритмы сжатия данных делятся на две большие категории:  
\begin{itemize}
	\item \textbf{С потерями} — применяются для мультимедиа (изображения, звук, видео), где допустимо частичное ухудшение качества ради сильного уменьшения объёма (JPEG, MP3, MPEG).  
	\item \textbf{Без потерь} — гарантируют точное восстановление исходных данных (LZW, Huffman, Deflate). Такие методы особенно важны для текстов, программ и других структурированных файлов.  
\end{itemize}

Алгоритм LZW относится к методам без потерь. Его основная идея заключается в построении динамического словаря повторяющихся подстрок, что позволяет эффективно кодировать данные с высокой степенью избыточности. Простота реализации и хорошая степень сжатия сделали его классическим примером практического применения теории информации.
