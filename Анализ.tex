% !TeX spellcheck = ru_RU-Russian
\section{Анализ предметной области}
\subsection{Характеристика области и её применения}

В современном цифровом мире безопасность информации является одной из важнейших задач. Практически любая сфера человеческой деятельности связана с обработкой и хранением данных: от личной переписки и банковских операций до управления промышленными системами и государственными структурами. Утечка или подделка информации может привести к серьёзным последствиям, поэтому возникает необходимость в надёжных методах защиты.  

Одним из ключевых инструментов обеспечения безопасности является шифрование. С помощью криптографических алгоритмов достигается защита данных от несанкционированного доступа, предотвращение подмены и обеспечение конфиденциальности. В современных условиях без применения криптографии невозможно представить работу интернета, электронных платежей, мобильной связи, облачных технологий и даже бытовых устройств, объединённых в сети «умного дома».  

\subsection{История и развитие симметричных алгоритмов шифрования}

Идея скрытия информации с помощью преобразований текста существует тысячелетиями. В древности применялись простейшие шифры, например, шифр Цезаря, в котором символы заменялись другими по фиксированному правилу. Несмотря на примитивность, такие методы долгое время использовались в дипломатии и военном деле.  

С развитием математики и вычислительной техники стали разрабатываться более сложные системы шифрования. Особое место среди них занимают симметричные блочные шифры, которые обрабатывают данные по частям (блоками) и используют один и тот же секретный ключ для операций шифрования и дешифрования.  

В 1977 году в США был принят стандарт DES (Data Encryption Standard). На протяжении двух десятилетий он широко применялся в коммерческих и государственных системах. Однако из-за короткой длины ключа (56 бит) и увеличения вычислительных возможностей DES постепенно утратил актуальность, так как стал подвержен атакам перебора.  

Для замены устаревшего DES в конце 90-х годов был проведён международный конкурс на разработку нового алгоритма шифрования. В 2000 году победителем стал алгоритм Rijndael, разработанный бельгийскими криптографами Винсентом Рейменом и Жоаном Дайменом. Он был стандартизирован в 2001 году под названием AES (Advanced Encryption Standard) и с тех пор является одним из основных мировых стандартов защиты информации.  

\subsection{Классификация методов шифрования}

Современные криптографические алгоритмы можно разделить на несколько классов.  

\begin{itemize}
	\item \textbf{Симметричные алгоритмы} — для шифрования и дешифрования используется один и тот же ключ. Они отличаются высокой скоростью работы и подходят для обработки больших массивов данных. К этой категории относятся алгоритмы AES, DES, 3DES, RC4.  
	\item \textbf{Асимметричные алгоритмы} — используют пару ключей: открытый для шифрования и закрытый для расшифровки. Такие методы сложнее в реализации и медленнее по скорости, но они обеспечивают более удобное распределение ключей. Примеры: RSA, ElGamal.  
	\item \textbf{Гибридные системы} — сочетают преимущества обоих подходов: асимметричное шифрование используется для безопасной передачи ключа, а симметричное — для быстрой работы с большими объёмами информации. Пример: протокол TLS, применяемый в защищённых интернет-соединениях (HTTPS).  
\end{itemize}

Алгоритм AES относится к симметричным блочным шифрам. Он работает с блоками фиксированной длины (128 бит) и может использовать ключи различной длины (128, 192 и 256 бит). Шифрование осуществляется в несколько раундов (10, 12 или 14 в зависимости от длины ключа), каждый из которых включает ряд преобразований: замену байтов по таблице подстановок, циклические сдвиги строк, линейное преобразование столбцов и наложение ключа.  

\subsection{Значение и применение AES}

На сегодняшний день AES является основным международным стандартом шифрования. Его надёжность подтверждена многолетними исследованиями криптографов, а высокая эффективность делает его подходящим для самых разных устройств — от мощных серверов до встроенных микроконтроллеров.  

Алгоритм AES применяется в следующих областях:  
\begin{itemize}
	\item защита интернет-соединений (HTTPS, VPN);  
	\item шифрование беспроводных сетей (WPA2, WPA3);  
	\item банковские системы и электронные платежи;  
	\item мобильные устройства и SIM-карты;  
	\item государственные стандарты и военные протоколы связи.  
\end{itemize}

Таким образом, AES стал универсальным инструментом обеспечения информационной безопасности, сочетающим в себе надёжность, производительность и широкую область применения. Его использование является необходимым условием для функционирования современных цифровых систем.  
